\documentclass{mcmthesis}
\mcmsetup{CTeX = false,  
        tcn = 2302952, problem = D,
        sheet = true, titleinsheet = true, keywordsinsheet = true,
        titlepage = false, abstract = false}
\usepackage{newtxtext}
\title{Wisdom Hidden in Teaming Strategies}
\date{\today}
\begin{document}
\begin{abstract}

\begin{keywords}
keyword1; keyword2
\end{keywords}
\end{abstract}
\maketitle
\tableofcontents
\newpage

\section{Introduction}
\subsection{Problem Background}
Nowadays, global demand for teamwork which are crucial for the development of the society has risen rapidly. As a competitive team sports, world football has ushered in a new hegemony without controversy. This let many football coach out of the cage to explore the methods of the winning game in the aspect of cooperation, especially in a typical team:Huskies. On the individual side, a star player may prompt the advantages of the teams into full play. However, despite the impacts of individual, the teamwork between teammates seem to be more significant to possess some good vibrations. Therefore, it is necessary to explore the rational teamwork strategies in a mathematically way and put forward some guides basing on the models for the Huskies coach.

\subsection{Problem Restatement}
\begin{itemize}
        \item Build a network which contains nodes and links for the ball passing to recognize the patterns and find out multiscales when observing the interaction.
        \item Determine the performance indicators reflected success and team level processes, and use them to construct a model to access the structure, configuration and dynamic of teamwork.
        \item Based on the teamwork model, tell the coach what structural strategies have been effective for the Huskies and what change can be made to promote success in the next competition.
        \item Discuss what other aspects of teamwork should be considered when generalizing the model of football teamwork to the team performance.
\end{itemize}
\section{Preparation of the Models}
\subsection{Assumptions}

\subsection{Notations}

\section{task 1}

\section{task 2}

\section{task 3}

\section{task 4}

\section{Strengths and weaknesses}

\subsection{Strengths}

\subsection{Weaknesses}

\newpage

\bibliographystyle{plain}
\bibliography{./reference/reference}

\begin{appendices}

\section{First appendix}

% Here are simulation programmes we used in our model as follow.\\

% \textbf{\textcolor[rgb]{0.98,0.00,0.00}{Input matlab source:}}
% \lstinputlisting[language=Matlab]{./code/mcmthesis-matlab1.m}

% \section{Second appendix}

% some more text \textcolor[rgb]{0.98,0.00,0.00}{\textbf{Input C++ source:}}
% \lstinputlisting[language=C++]{./code/mcmthesis-sudoku.cpp}

\end{appendices}
\end{document}
